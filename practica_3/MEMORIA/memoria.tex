\documentclass[12pt]{report}
\usepackage[spanish]{babel}
\usepackage{float}
\usepackage{listings}
%\usepackage{natbib}
\usepackage{url}
\usepackage[utf8x]{inputenc}
\usepackage{amsmath}
\usepackage{graphicx}
\graphicspath{{images/}}
\usepackage{parskip}
\usepackage{fancyhdr}
\usepackage{booktabs}
\usepackage{vmargin}
\setmarginsrb{3 cm}{2.6 cm}{3 cm}{2.5 cm}{1 cm}{1.5 cm}{1 cm}{1.5 cm}
\usepackage{color}
\definecolor{mygreen}{rgb}{0,0.6,0}
\definecolor{mygray}{rgb}{0.5,0.5,0.5}
\definecolor{mymauve}{rgb}{0.58,0,0.82}
 
\lstset{ %
backgroundcolor=\color{white},
basicstyle=\footnotesize, 
breakatwhitespace=false, 
breaklines=true,
captionpos=b,
commentstyle=\color{mygreen},
deletekeywords={...}, 
escapeinside={\%*}{*)}, % if you want to add LaTeX within your code
extendedchars=true, 
frame=single, % adds a frame around the code
keepspaces=true, 
keywordstyle=\color{blue}, % keyword style
% language=Octave, % the language of the code
morekeywords={*,...}, 
numbers=left, 
numbersep=5pt, 
numberstyle=\tiny\color{mygray},
rulecolor=\color{black}, 
showspaces=false, 
showstringspaces=false, % underline spaces within strings only
showtabs=false, % show tabs within strings adding particular underscores
stepnumber=1, % the step between two line-numbers. If it's 1, each line will be numbered
stringstyle=\color{mymauve}, % string literal style
tabsize=2, % sets default tabsize to 2 spaces
title=\lstname 
}
%END of listing package%
 
\definecolor{darkgray}{rgb}{.4,.4,.4}
\definecolor{purple}{rgb}{0.65, 0.12, 0.82}
 
%define Javascript language
\lstdefinelanguage{JavaScript}{
keywords={typeof, new, true, false, catch, function, return, null, catch, switch, var, if, in, while, do, else, case, break},
keywordstyle=\color{blue}\bfseries,
ndkeywords={class, export, boolean, throw, implements, import, this},
ndkeywordstyle=\color{darkgray}\bfseries,
identifierstyle=\color{black},
sensitive=false,
comment=[l]{//},
morecomment=[s]{/*}{*/},
commentstyle=\color{purple}\ttfamily,
stringstyle=\color{red}\ttfamily,
morestring=[b]',
morestring=[b]"
}
 

\title{Nuestro Proyecto Final - UniTools}					
\usepackage{xcolor}

\definecolor{codegreen}{rgb}{0,0.6,0}
\definecolor{codegray}{rgb}{0.5,0.5,0.5}
\definecolor{codepurple}{rgb}{0.58,0,0.82}
\definecolor{backcolour}{rgb}{0.95,0.95,0.92}

\lstdefinestyle{mystyle}{
    backgroundcolor=\color{backcolour},   
    commentstyle=\color{codegreen},
    keywordstyle=\color{magenta},
    numberstyle=\tiny\color{codegray},
    stringstyle=\color{codepurple},
    basicstyle=\ttfamily\footnotesize,
    breakatwhitespace=false,         
    breaklines=true,                 
    captionpos=b,                    
    keepspaces=true,                 
    numbers=left,                    
    numbersep=5pt,                  
    showspaces=false,                
    showstringspaces=false,
    showtabs=false,                  
    tabsize=2
}

\lstset{style=mystyle}


% Title
\author{- A.W.}						
% Author
\date{today's date}
% Date

\makeatletter
\let\thetitle\@title
\let\theauthor\@author
\let\thedate\@date
\makeatother

\pagestyle{fancy}
\fancyhf{}
\rhead{\theauthor}
\lhead{\thetitle}
\cfoot{\thepage}
%%%%%%%%%%%%%%%%%%%%%%%%%%%%%%%%%%%%%%%%%%%%
\begin{document}

%%%%%%%%%%%%%%%%%%%%%%%%%%%%%%%%%%%%%%%%%%%%%%%%%%%%%%%%%%%%%%%%%%%%%%%%%%%%%%%%%%%%%%%%%

\begin{titlepage}
	\centering
    \vspace*{0.5 cm}
   % \includegraphics[scale = 0.075]{bsulogo.png}\\[1.0 cm]	% University Logo
\begin{center}    \textsc{\Large  APLICACIONES WEB}\\[2.0 cm]	\end{center}% University Name
	\textsc{\Large Memoria del Proyecto }\\[0.5 cm]				% Course Code
	\rule{\linewidth}{0.2 mm} \\[0.4 cm]
	{ \huge \bfseries \thetitle}\\
	\rule{\linewidth}{0.2 mm} \\[1.5 cm]
	
	\begin{minipage}{0.4\textwidth}
		\begin{flushleft} \large
		%	\emph{Submitted To:}\\
		%	Name\\
          % Affiliation\\
           %contact info\\
			\end{flushleft}
			\end{minipage}~
			\begin{minipage}{0.4\textwidth}
            
			\begin{flushright} \large
			\emph{Miembros del grupo :} \\
			Luis Cepeda, Carlos Bilbao,
			Hugo Ribeiro, Daniel Canseco,
			Fernando Ruiz, Bruno Mayo
			
		\end{flushright}
           
	\end{minipage}\\[2 cm]
	
	\includegraphics[scale = 0.15]{UniTools.png}

\end{titlepage}

\tableofcontents
\pagebreak

\renewcommand{\thesection}{\arabic{section}}
\section{Introducción a UniTools y Motivación}

Nuestro proyecto ha consistido en el desarrollo de la aplicación Web \textit{UniTools}, una plataforma on-line de gestión de código, centrado en los alumnos de la Facultad de Informática de la UCM. Por tanto, los potenciales usuarios de nuestro producto son nuestros compañeros, estudiantes de grados que requieren el uso de código fuente.

La web sirve para almacenar versiones de código de los proyectos de los estudiantes, de manera privada si así lo desean, y además ofrece una serie de herramientas que pueden ser de ayuda a la hora de programar. Para ayudar a \textbf{gestionar Software}, UniTools ofrece funcionalidades como Privacidad, subida y bajada de archivos, un sistema de valoración de los Proyectos de cero a cinco estrellas, y un sistema de mutexes Candado para evitar la edición en paralelo del código, lo que generaría conflictos, entre otras funcionalidades.

Una vez un usuario crea un Proyecto, puede verlo desde su propio Perfil, donde además puede \textbf{actualizar su foto de perfil} además del resto de sus datos. Entre estos, se encuentra la contraseña que se almacena cifrada en la Base de Datos. Esta foto de pefil aparece también cuando el usuario \textbf{responde a un post del Foro}, el cual incluye además una barra de búsqueda para facilitar su uso. Este foro, que es una manera de que los estudiantes se ayuden entre ellos a resolver dudas, cuenta con el apoyo de los \textbf{administradores}, usuarios con un rol especial que les confiere mayores privilegios.

Todo esto lo hemos querido desarrollar sin dejar de tener en cuenta la \textbf{experiencia de usuario}, por lo que hemos dedicado mucho tiempo a conseguir un diseño completo gracias al CSS (con \textbf{Flex y Grid}), así como en añadir detalles como la música de fondo, y la posibilidad de personalizar la experiencia por parte del usuario cambiando entre el modo Dark (colores invertidos) y el normal. Igualmente, este proyecto se ha hecho además teniendo en cuenta las guías de la asignatura para un proyecto bien organizado (estructurado) y con buenas prácticas de programación web, como es el uso del \textbf{patrón DAO/Transfer Object} y el uso de \textbf{Ajax}, migrando todo finalmente a un VPS.

La primera práctica que entregamos fue solo el comienzo de la web, mientras que la siguiente supuso un importante avance ya que organizamos nuestra página web haciendo uso de un esquema que dividía en cuatro partes (cabecera, navegación ó menú, contenido y pie), siendo el contenido el único que era actualizado dinámicamente, en función de las peticiones recbidas. En esa misma práctica comenzamos a desarrollar funcionalidades de Login y Registro haciendo uso de DAO así como Herramientas. La tercera práctica supuso el cambio más drástico, ya que conseguimos terminar el Foro, comenzar funcionalidades de Proyectos y de Perfil y añadir mejoras tanto en la Base de Datos como en los script, incluidos los Form. 

Finalmente, en esta entrega del Proyecto Final hemos acabado de pulir el CSS, terminado las funcionalidades de Proyectos, hemos añadido otras tecnologías como Ajax, mejorado la Base de Datos y los script e incluido otras mejoras puntuales.

En conjunto, estamos muy satisfechos con el resultado final, que si bien ha requerido trabajo, ha resultado muy instructivo. Hemos hecho todo lo posible para que esta Memoria refleje nuestro esfuerzo, describiendo la arquitectura del proyecto, ofreciendo instrucciones de instalación y capturas de las vistas de la página web a modo de Manual de Uso, entre otras explicaciones. 

\section{Tour: Guía de Navegación}

// TODO

\section{Descripción de la arquitectura, JS, PHP y vistas}

// TODO

\section{Descripción de la Base de Datos}

// TODO

\section{Instrucciones de instalación}

// TODO

\section{Descripción de trabajo y aportes individuales}

// TODO

\begin{table}[h]
\begin{tabular}{@{}|l|l|@{}}
\toprule
\textbf{Miembro} & \textbf{Actividades}                                                                                                                                   \\ \midrule
Luis Cepeda      & Dirigido toda la mejora del CSS.                                                                                                                            \\ \midrule
Carlos Bilbao    & \begin{tabular}[c]{@{}l@{}}Redactado la Memoria. Ha ayudado con lo relacionado con Proyectos.\\  Hizo junto a Bruno lo relativo a Forms.\end{tabular}   \\ \midrule
Hugo Ribeiro     & Ayudado con la BBDD, el CSS y mucho con funcionalidad en general.                                                                                               \\ \midrule
Daniel Canseco   & \begin{tabular}[c]{@{}l@{}}Creado la página Team, añadido el modo Dark, \\ la música y ayudado con el CSS.\end{tabular}                                \\ \midrule
Fernando Ruiz   & \begin{tabular}[c]{@{}l@{}}Añadido la funcionalidad de edición de perfil y \\ la barra de búsqueda. Ayudo con el CSS.\end{tabular}                                     \\ \midrule
Bruno Mayo       & \begin{tabular}[c]{@{}l@{}}Mejorado la Base de Datos. Ha ayudado a crear las \\ clases FormLogin, FormRegistro y Form y a implementarlas.\end{tabular} \\ \bottomrule
\end{tabular}
\end{table}

\end{document}
