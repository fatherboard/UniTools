\documentclass[12pt]{report}
\usepackage[spanish]{babel}
\usepackage{float}
\usepackage{listings}
%\usepackage{natbib}
\usepackage{url}
\usepackage[utf8x]{inputenc}
\usepackage{url}
\usepackage{hyperref}
\hypersetup{
    colorlinks = true,
    linkbordercolor = {white},
    linkcolor = blue,
}
\usepackage{amsmath}
\usepackage{graphicx}
\graphicspath{{images/}}
\usepackage{parskip}
\usepackage{fancyhdr}
\usepackage{booktabs}
\usepackage{vmargin}
\setmarginsrb{3 cm}{2.6 cm}{3 cm}{2.5 cm}{1 cm}{1.5 cm}{1 cm}{1.5 cm}
\usepackage{color}
\definecolor{mygreen}{rgb}{0,0.6,0}
\definecolor{mygray}{rgb}{0.5,0.5,0.5}
\definecolor{mymauve}{rgb}{0.58,0,0.82}
 
\lstset{ %
backgroundcolor=\color{white},
basicstyle=\footnotesize, 
breakatwhitespace=false, 
breaklines=true,
captionpos=b,
commentstyle=\color{mygreen},
deletekeywords={...}, 
escapeinside={\%*}{*)}, % if you want to add LaTeX within your code
extendedchars=true, 
frame=single, % adds a frame around the code
keepspaces=true, 
keywordstyle=\color{blue}, % keyword style
% language=Octave, % the language of the code
morekeywords={*,...}, 
numbers=left, 
numbersep=5pt, 
numberstyle=\tiny\color{mygray},
rulecolor=\color{black}, 
showspaces=false, 
showstringspaces=false, % underline spaces within strings only
showtabs=false, % show tabs within strings adding particular underscores
stepnumber=1, % the step between two line-numbers. If it's 1, each line will be numbered
stringstyle=\color{mymauve}, % string literal style
tabsize=2, % sets default tabsize to 2 spaces
title=\lstname 
}
%END of listing package%
 
\definecolor{darkgray}{rgb}{.4,.4,.4}
\definecolor{purple}{rgb}{0.65, 0.12, 0.82}
 
%define Javascript language
\lstdefinelanguage{JavaScript}{
keywords={typeof, new, true, false, catch, function, return, null, catch, switch, var, if, in, while, do, else, case, break},
keywordstyle=\color{blue}\bfseries,
ndkeywords={class, export, boolean, throw, implements, import, this},
ndkeywordstyle=\color{darkgray}\bfseries,
identifierstyle=\color{black},
sensitive=false,
comment=[l]{//},
morecomment=[s]{/*}{*/},
commentstyle=\color{purple}\ttfamily,
stringstyle=\color{red}\ttfamily,
morestring=[b]',
morestring=[b]"
}
 

\title{Nuestro Proyecto Final - UniTools}					
\usepackage{xcolor}

\definecolor{codegreen}{rgb}{0,0.6,0}
\definecolor{codegray}{rgb}{0.5,0.5,0.5}
\definecolor{codepurple}{rgb}{0.58,0,0.82}
\definecolor{backcolour}{rgb}{0.95,0.95,0.92}

\lstdefinestyle{mystyle}{
    backgroundcolor=\color{backcolour},   
    commentstyle=\color{codegreen},
    keywordstyle=\color{magenta},
    numberstyle=\tiny\color{codegray},
    stringstyle=\color{codepurple},
    basicstyle=\ttfamily\footnotesize,
    breakatwhitespace=false,         
    breaklines=true,                 
    captionpos=b,                    
    keepspaces=true,                 
    numbers=left,                    
    numbersep=5pt,                  
    showspaces=false,                
    showstringspaces=false,
    showtabs=false,                  
    tabsize=2
}

\lstset{style=mystyle}


% Title
\author{- A.W.}						
% Author
\date{today's date}
% Date

\makeatletter
\let\thetitle\@title
\let\theauthor\@author
\let\thedate\@date
\makeatother

\pagestyle{fancy}
\fancyhf{}
\rhead{\theauthor}
\lhead{\thetitle}
\cfoot{\thepage}
%%%%%%%%%%%%%%%%%%%%%%%%%%%%%%%%%%%%%%%%%%%%
\begin{document}

%%%%%%%%%%%%%%%%%%%%%%%%%%%%%%%%%%%%%%%%%%%%%%%%%%%%%%%%%%%%%%%%%%%%%%%%%%%%%%%%%%%%%%%%%

\begin{titlepage}
	\centering
    \vspace*{0.5 cm}
   % \includegraphics[scale = 0.075]{bsulogo.png}\\[1.0 cm]	% University Logo
\begin{center}    \textsc{\Large  APLICACIONES WEB}\\[2.0 cm]	\end{center}% University Name
	\textsc{\Large Memoria del Proyecto }\\[0.5 cm]				% Course Code
	\rule{\linewidth}{0.2 mm} \\[0.4 cm]
	{ \huge \bfseries \thetitle}\\
	\rule{\linewidth}{0.2 mm} \\[1.5 cm]
	
	\begin{minipage}{0.4\textwidth}
		\begin{flushleft} \large
		%	\emph{Submitted To:}\\
		%	Name\\
          % Affiliation\\
           %contact info\\
			\end{flushleft}
			\end{minipage}~
			\begin{minipage}{0.4\textwidth}
            
			\begin{flushright} \large
			\emph{Miembros del grupo :} \\
			Luis Cepeda, Carlos Bilbao,
			Hugo Ribeiro, Daniel Canseco,
			Fernando Ruiz, Bruno Mayo
			
		\end{flushright}
           
	\end{minipage}\\[2 cm]
	
	\includegraphics[scale = 0.15]{UniTools.png}

\end{titlepage}

\tableofcontents
\pagebreak

\renewcommand{\thesection}{\arabic{section}}
\section{Introducción a UniTools y Motivación}

Nuestro proyecto ha consistido en el desarrollo de la aplicación Web \textit{UniTools}, una plataforma on-line de gestión de código, centrado en los alumnos de la Facultad de Informática de la UCM. Por tanto, los potenciales usuarios de nuestro producto son nuestros compañeros, estudiantes de grados que requieren el uso de código fuente.

La web sirve para almacenar versiones de código de los proyectos de los estudiantes, de manera privada si así lo desean, y además ofrece una serie de herramientas que pueden ser de ayuda a la hora de programar. Para ayudar a \textbf{gestionar Software}, UniTools ofrece funcionalidades como Privacidad, subida y bajada de archivos, un sistema de valoración de los Proyectos de cero a cinco estrellas, y un sistema de mutexes Candado para evitar la edición en paralelo del código, lo que generaría conflictos, entre otras funcionalidades.

Una vez un usuario crea un Proyecto, puede verlo desde su propio Perfil, donde además puede \textbf{actualizar su foto de perfil} además del resto de sus datos. Entre estos, se encuentra la contraseña que se almacena cifrada en la Base de Datos. Esta foto de pefil aparece también cuando el usuario \textbf{responde a un post del Foro}, el cual incluye además una barra de búsqueda para facilitar su uso. Este foro, que es una manera de que los estudiantes se ayuden entre ellos a resolver dudas, cuenta con el apoyo de los \textbf{administradores}, usuarios con un rol especial que les confiere mayores privilegios.

Todo esto lo hemos querido desarrollar sin dejar de tener en cuenta la \textbf{experiencia de usuario}, por lo que hemos dedicado mucho tiempo a conseguir un diseño completo gracias al CSS (con \textbf{Flex y Grid}), así como en añadir detalles como la música de fondo, y la posibilidad de personalizar la experiencia por parte del usuario cambiando entre el modo Dark (colores invertidos) y el normal. Igualmente, este proyecto se ha hecho además teniendo en cuenta las guías de la asignatura para un proyecto bien organizado (estructurado) y con buenas prácticas de programación web, como es el uso del \textbf{patrón DAO/Transfer Object} y el uso de \textbf{Ajax}, migrando todo finalmente a un VPS.

La primera práctica que entregamos fue solo el comienzo de la web, mientras que la siguiente supuso un importante avance ya que organizamos nuestra página web haciendo uso de un esquema que dividía en cuatro partes (cabecera, navegación ó menú, contenido y pie), siendo el contenido el único que era actualizado dinámicamente, en función de las peticiones recibidas. En esa misma práctica comenzamos a desarrollar funcionalidades de Login y Registro haciendo uso de DAO así como Herramientas. La tercera práctica supuso el cambio más drástico, ya que conseguimos terminar el Foro, comenzar funcionalidades de Proyectos y de Perfil y añadir mejoras tanto en la Base de Datos como en los script, incluidos los Form. 

Finalmente, en esta entrega del Proyecto Final hemos acabado de pulir el CSS, terminado las funcionalidades de Proyectos, hemos añadido otras tecnologías como Ajax, mejorado la Base de Datos y los script e incluido otras mejoras puntuales.

En conjunto, estamos muy satisfechos con el resultado final, que si bien ha requerido trabajo, ha resultado muy instructivo. Hemos hecho todo lo posible para que esta Memoria refleje nuestro esfuerzo, describiendo la arquitectura del proyecto, ofreciendo instrucciones de instalación y capturas de las vistas de la página web a modo de Manual de Uso, entre otras explicaciones. 


\subsection{Posible Future Work}

Uno de los objetivos de esta memoria era ofrecer un documento tal y como "se lo venderíamos a una empresa". Por tanto, nos pareció buena idea añadir una sección dedicada a ideas de futuro con las que nos gustaría continuar mejorando el Proyecto, en el caso de que esta no hubiera sido la última práctica. Esto habría sido buena idea para una empresa ya que nos habría servido para demostrar la calidad de nuestras ideas.

Siendo sinceros, nos apena un poco que esta sea la última entrega, en el sentido de que ha sido la primera en la que hemos podido empezar a implementar funcionalidades que eran verdaderamente parecidas a la manera en que queríamos que acabar esta web: Como una manera de gestionar proyectos entre compañeros. Es decir, hemos podido empezar a ver resultados más allá de una web básica.

Si bien es cierto que hemos podido añadir las funcionalidades principales, tales como la valoración de los Proyectos, su política de privacidad, o la subida o bajada de archivos, aún podríamos añadir muchas más en el futuro. Estas podrían por ejemplo ser:

\begin{itemize}
    \item Un editor de texto en línea para el código subido por los estudiantes.
    \item Un detector automático del lenguaje de programación del Proyecto en función de la extensión mayoritaria de sus archivos.
    \item Un nuevo rol, de Profesor, desde el que se hubiera podido calificar las entregas de los estudiantes, sus Proyectos.
    \item Un sistema de "fork" de Proyectos por el cual un alumno pudiera copiar todo el repositorio de otros para seguir desarrollándolo en otra dirección.
    \item Estadísticas dentro del Proyecto para poder comprobar qué alumno ha contribuido más (por ejemplo, mostrando la cantidad de archivos subidos por cada uno de los componentes del grupo).
\end{itemize}

Al mismo tiempo se podrían en el futuro añadir otras funcionalidades menos relevantes, pero que mejorarían la experiencia de usuario, como una sección de Mensajes Privados, un Calendario con tareas pendientes y estadísticas en el Perfil del usuario (archivos subidos, media de las valoraciones en estrellas de sus Proyectos...). Otra funcionalidad extra que nos habría gustado añadir en el futuro es la posibilidad de conectarse mediante terminal directamente con UniTools (es decir, un binario de Linux que estableciera conexión con el servidor, intercambiara credenciales y permitiera al usuario actualizar sus Proyectos, al estilo de \path{git}).

\section{Tour: Guía de Navegación}
\subsection{Mejoras finales}
Hemos introducido multitud de mejoras desde la última entrega de este proyecto, no solo mejorando la apariencia (CSS, Flex y Grid) sino también con nuevas funcionalidades. Antes de comenzar a mostrar capturas de pantalla de distintas partes de la web, nos gustaría enumerar las modificaciones más importantes, evitando así que pudieran pasar desapercibidas:
\begin{enumerate}
    \item Hemos añadido un \textbf{sistema de permisos a los Proyectos}. En concreto, los proyectos otorgan tres niveles de privilegios. El nivel 0 es el del Creador, el nivel 1 es de aquellos con permiso de lectura y el nivel 2 para aquellos con permiso de lectura y escritura.
    
    En general, la funcionalidad de Proyectos se ha extendido en varios sentidos.
    
    \item Hemos implementado validación de los valores del registro mediante \textbf{jQuery y Ajax}. Como se puede apreciar en el código, no nos hemos limitado a una validación básica.
    
    \item Hemos mejorado tanto la seguridad como la experiencia de usuario. Hemos añadido un \textbf{pop-up interactivo} que permite al usuario actualizar a la versión Premium (con el botón de Perfil). En el perfil el usuario puede además ver sus proyectos. Estos son ejemplos de algunas de las mejoras de esta versión.
    
    \item Añadir funcionalidad ha requerido mejorar la Base de Datos también. En particular, ha sido necesario añadir una nueva tabla para los permisos, aunque también se han hecho modificaciones para permitir \textbf{a los alumnos valorar los Proyectos}, mediante un sistema de estrellas.
\end{enumerate}
\subsection{Manual con capturas}
El usuario comienza viendo la página de Inicio. Existe el modo normal y el modo Dark, que invierte los colores. Puede a continuación registrarse. 

\begin{figure}[h]
 \centering
  \includegraphics[width=0.95\linewidth]{2.jpg}
\end{figure}

\begin{figure}[!h]
 \centering
  \includegraphics[width=0.4144\linewidth]{3.jpeg}
\end{figure}

El registro valida simultáneamente, mediante jQuery, los campos introducidos y muestra iconos de alerta o confirmación según sea el caso, con mensajes explicativos. Tras el registro, el usuario podrá plantear sus dudas a modo de post en el Foro, así como contestar las de sus compañeros y las respuestas de estos. 

\begin{figure}[!h]
 \centering
  \includegraphics[width=1.01\linewidth]{6.jpg}
\end{figure}

Como se puede ver, \textbf{el Foro cuenta con un buscador}. Entremos en un Post para leer las respuestas, anidadas cuando se responden entre ellas.

\begin{figure}[!h]
 \centering
  \includegraphics[width=0.9\linewidth]{7.jpg}
\end{figure}

En el foro podrá ver las fotos de perfil de sus compañeros, el también podrá subir una suya si así lo desea desde el Perfil. Desde allí podrá actualizar el restos de sus datos. \textbf{Abajo a la derecha aparece el botón de pasar a Premium.} En el perfil el usuario puede gestionar sus proyectos.

\begin{figure}[!h]
 \centering
  \includegraphics[width=\linewidth]{4.jpg}
\end{figure}

También desde Perfil podrá actualizar su versión a Premium, si desea disfrutar de ventajas como proyectos privados.

\begin{figure}[!h]
 \centering
  \includegraphics[width=0.65\linewidth]{13.jpg}
\end{figure}

Llegada la hora, podrá crear su propio proyecto en Proyectos. Usando el formulario de creación (FormNewProject, que hereda de la clase Form, como otros) donde podrá especificar el Contenido y el lenguaje de programación entre otros.

\begin{figure}[!h]
 \centering
  \includegraphics[width=0.4\linewidth]{9.jpeg}
\end{figure}

Una vez creado el Proyecto, el usuario puede subir archivos, y eliminarlos más tarde si así lo desea tras confirmar. 

\begin{figure}[!h]
 \centering
  \includegraphics[width=1.1\linewidth]{10.jpg}
\end{figure}

Los Proyectos además se pueden valorar.

\begin{figure}[!h]
 \centering
  \includegraphics[width=0.6\linewidth]{12.jpg}
\end{figure}

Entremos en un Proyecto. En ellos, existen multitud de acciones posibles como añadir permisos de lectura y/o escritura a otros usuarios.

\begin{figure}[!h]
 \centering
  \includegraphics[width=1.15\linewidth]{20.jpeg}
\end{figure}

Un último detalle que merece la pena mencionar es que hemos añadido un reproductor de audio por si el usuario quisiera pausar la música que se reproduce al Inicio.

\begin{figure}[!h]
 \centering
  \includegraphics[width=0.6\linewidth]{21.jpeg}
\end{figure}


\section{Descripción de la arquitectura, JavaScript, PHP y vistas}

Para entender la arquitectura de la página web primero hay que echar un vistazo al código de \path{/index.php}. En él se distinguen las partes de Navegación, Cabecera y Contenido. Esta estructura se repite continuamente, cambiando el contenido según cual sea la petición del usuario, gestionada en \path{includes/common/contenido.php}. Los accesos a lugares indebidos son gestionados por \path{/includes/.htaccess}, añadiendo seguridad a la web y utilizando los mensajes de error del directorio \path{/err}.

En cuanto al funcionamiento de casi todo lo que ofrece lo web, ha sido realizado buscando seguir los patrones de diseño adecuados y las abstracciones más correctas. Esto en general ha supuesto \textbf{el uso del patrón DAO} para comunicarse con la Base de Datos, utilizando Transfer Objects (TOUs) como vía de intercambio de información.

Por otro lado, se han usado clases que permitieran reducir el código. Por ejemplo, \textbf{todos los formularios de la web} (para el Login, para el Registro, para crear un nuevo proyecto,...) \textbf{heredan de la clase Form}, alojada en \path{/includes/Form.php}.

Por citar un ejemplo de uso que englobe todo esto, cuando un usuario se registra está utilizando unos campos generados por un \textit{new FormRegistro()} invocado desde \path{/registro.php}. Si no se encuentran errores de validación, se creará un objeto \textit{new DAOUsuario()}, con el que se insertará el nuevo usuario en la Base de Datos pasándole un \textit{new TOUser()} con la información que el usuario haya introducido.

El proyecto también tiene código en JavaScript. Todas las herramientas de la sección Herramientas usan de hecho ese lenguaje, así como la validación de los datos del usuario, que se consigue con \textbf{Ajax y jQuery}, desde la carpeta \path{/js}.

Cuando el usuario accede a la web en Inicio, escucha la música en mp3 almacenada en \path{practica_3/music/}. Esto lo hacemos con la etiqueta \textit{embed}. Esto es solo un ejemplo de los toques que hemos añadido a la web para hacerla más personal, aunque gran parte de esto se consigue con el CSS, almacenado en el directorio \path{/css}. Se trata de una de las partes más elaboradas. Tanto es así, que algunas partes cuentan con su propia hoja de estilo, como el pop-up de premium (en \path{/css/popup-prem.css}).

Este pop-up, que utiliza el script con Modal Box, se puede encontrar en \path{/script/modal.js}, también en JavaScript. Se pueden encontrar muchos otros scripts en el proyecto, como \path{/uploadPhoto.php} para actualizar la foto de perfil del usuario o \path{/search.php} para la búsqueda entre posts.

\section{Descripción de la Base de Datos}

La Base de Datos ha ido creciendo con las sucesivas versiones de la web, y actualmente cuenta con multitud de tablas y relaciones entre sus campos. Existe una tabla SQL para representar a los usuarios, otra para las entradas del Foro (\textit{posts}), para cada proyecto (\textit{project}),  respuesta, y desde esta versión también \textit{permissions} para \textbf{los permisos del Proyecto}. Se ha creado además otra tabla \textbf{estrellas} para las valoraciones de los usuarios de los Proyectos.

La tabla del usuario no contiene solo la información imprescindible, sino también otra como \textit{aboutMe}, para permitir a los usuarios personalizar su perfil.

Para poder dar una imagen global de las tablas, mostramos a continuación todas las tablas con sus atributos y sin sus relaciones.

\begin{figure}[!h]
 \centering
  \includegraphics[width=\linewidth]{30.jpg}
\end{figure}

La tabla de los proyectos es una de las más importantes, de modo que sería conveniente describirla brevemente aquí.

\begin{lstlisting}[language=SQL]
CREATE TABLE `project` (
  `id` int(11) NOT NULL,
  `titulo` varchar(100) NOT NULL,
  `candado` tinyint(1) NOT NULL,
  `userId` int(11) NOT NULL,
  `privado` tinyint(1) NOT NULL,
  `lenguaje` varchar(50) NOT NULL,
  `contenido` varchar(200) NOT NULL
) ENGINE=InnoDB DEFAULT CHARSET=utf8mb4;

\end{lstlisting}

El \textit{id}, es el identificador único. El \textit{título}, el que se elija para el Proyecto. El \textit{Candado} se utiliza para evitar problemas de conflictos modificando código simultáneamente (a modo de Mutex), las \textit{estrellas} se utilizan para que los usuarios puedan valorar los proyectos de sus compañeros, \textit{privado} es un booleano que puede ser modificado por los usuarios premium, \textit{lenguaje} se refiere al lenguaje de programación principal del proyecto y \textit{contenido} permite al usuario añadir una descripción de la temática del proyecto. Cabe resaltar que \textbf{esta tabla ya no tiene un campo estrellas, ya que eso se gestiona con la nueva tabla \textit{estrellas}.}


\section{Instrucciones de instalación}

Si el alumno tiene una distribución GNU/Linux, puede acceder en local a la web clonando el repositorio y almacenándolo en \path{/var/www/html}. Esto puede hacerse con el programa \textit{git}, en cuanto hagamos público el código.
\newline

\begin{lstlisting}[language=SQL]
$ git clone https://github.com/fatherboard/UniTools.git
$ sudo -s && cd UniTools
# mv * /var/www/html
\end{lstlisting}
Eso sí, el usuario deberá tener antes instaladas las dependencias, que son \textit{mysql}, y según cual sea la distribución, \textit{apache} y \textit{phpmyadmin}. La instalación de estos programas dependerá del gestor de paquetes que utilice el sistema. Asumiendo el de Ubuntu o Debian \textit{apt} (probablemente el más común) sería:
\newline
\begin{lstlisting}[language=SQL]
$ sudo apt update
$ sudo apt install mysql phpmyadmin php-mbstring php-gettext
\end{lstlisting}
Para sistemas Windows, se recomienda instalar la famosa herramienta open-source Xampp, desde la \hyperlink{https://www.apachefriends.org/download.html}{página de descarga}. Una vez se tenga el .exe, se puede instalar sin opciones de configuración diferentes de las por defecto. El proyecto deberá ponerse en la carpeta \path{.htdocs}, localizada en el disco principal del sistema. Tras eso, bastará con iniciar el Apache y el SQL.

La Base de Datos cuenta con información de ejemplo, y deberá importarse a \textit{Php-MyAdmin}. El usuario puede encontrar el archivo .sql en el directorio \path{/mysql}.

Es decir, para hacer funcionar la web con Xampp es imprescindible importar los archivos \path{/mysql/unitoolsdb.sql} y \path{/mysql/usuariosbd.sql} a una base de datos llamada UnitoolsDB en Phpmyadmin.

Existe un \textbf{usuario de prueba con nombre \textit{bruno} con contraseña 1}. También se puede adicionalmente entrar desde los usuarios \textit{carlos, fer, luis, hugo} y \textit{cansek} con la misma contraseña.

\newpage

\section{Descripción de trabajo y aportes individuales}

Aunque a continuación se muestre la principal contribución de cada miembro del equipo a esta entrega, todos hemos colaborado un poco en todo lo que compone el proyecto: Es decir, todos hemos aprendido JavaScript, utilizado DAOs, practicado con Flex y Grid, etc... en mayor o menor medida.

\begin{table}[h]
\begin{tabular}{@{}|l|l|@{}}
\toprule
\textbf{Miembro} & \textbf{Actividades}                                                                                                                                   \\ \midrule
Luis Cepeda      & Mejorado el CSS y los script.                                                                                                                            \\ \midrule
Carlos Bilbao    & \begin{tabular}[c]{@{}l@{}}Redactado la Memoria. Ha ayudado con la mejora de los script.\end{tabular}   \\ \midrule
Hugo Ribeiro     & Mejorado funcionalidad de Proyectos significativamente.                                                                                  \\ \midrule
Daniel Canseco   & \begin{tabular}[c]{@{}l@{}}Se centró en el jQuery y el Ajax.\end{tabular}                                \\ \midrule
Fernando Ruiz   & \begin{tabular}[c]{@{}l@{}}Añadido la funcionalidad de pop-up premium y ayuda en scripts. \end{tabular}                                     \\ \midrule
Bruno Mayo       & \begin{tabular}[c]{@{}l@{}}Mejorado la Base de Datos y el CSS.\end{tabular} \\ \bottomrule
\end{tabular}
\end{table}

\end{document}
